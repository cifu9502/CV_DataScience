%%%%%%%%%%%%%%%%%%%%%%%%%%%%%%%%%%%%%%%%%
% "ModernCV" CV and Cover Letter
% LaTeX Template
% Version 1.11 (19/6/14)
%
% This template has been downloaded from:
% http://www.LaTeXTemplates.com
%
% Original author:
% Xavier Danaux (xdanaux@gmail.com)
%
% License:
% CC BY-NC-SA 3.0 (http://creativecommons.org/licenses/by-nc-sa/3.0/)
%
% Important note:
% This template requires the moderncv.cls and .sty files to be in the same 
% directory as this .tex file. These files provide the resume style and themes 
% used for structuring the document.
%
%%%%%%%%%%%%%%%%%%%%%%%%%%%%%%%%%%%%%%%%%

%----------------------------------------------------------------------------------------
%	PACKAGES AND OTHER DOCUMENT CONFIGURATIONS
%----------------------------------------------------------------------------------------

\documentclass[10pt,a4paper,sans]{moderncv} % Font sizes: 10, 11, or 12; paper sizes: a4paper, letterpaper, a5paper, legalpaper, executivepaper or landscape; font families: sans or roman

\moderncvstyle{classic} % CV theme - options include: 'casual' (default), 'classic', 'oldstyle' and 'banking'
\moderncvcolor{blue} % CV color - options include: 'blue' (default), 'orange', 'green', 'red', 'purple', 'grey' and 'black'

\usepackage{lipsum} % Used for inserting dummy 'Lorem ipsum' text into the template

\usepackage{multicol}
\usepackage[scale=0.75,top=2cm,bottom=2.7cm]{geometry} % Reduce document margins
\setlength{\hintscolumnwidth}{2.3cm} % Uncomment to change the width of the dates column
%\setlength{\makecvtitlenamewidth}{10cm} % For the 'classic' style, uncomment to adjust the width of the space allocated to your name
\usepackage{footmisc}



%----------------------------------------------------------------------------------------
%	NAME AND CONTACT INFORMATION SECTION
%----------------------------------------------------------------------------------------

\firstname{{\huge Jes\'us David}} % Your first name
\familyname{{\huge Cifuentes Pardo} } % Your last name

% All information in this block is optional, comment out any lines you don't need
\title{Curriculum Vitae  }
\address{Calle 137A No 58-35 \\ Int 3 Ap 304\\}{Bogot\'a, Colombia}


\mobile{  +57 3105675255}
\email{jesuscif@if.usp.br}
\photo[80pt][1.3pt]{foto2.JPG}  

%----------------------------------------------------------------------------------------

\begin{document}


\makecvtitle % Print the CV title
%----------------------------------------------------------------------------------------
%	EDUCATION SECTION
%----------------------------------------------------------------------------------------


%\begin{picture}(0,0)
%\put(300,10){\includegraphics[scale=0.25]{foto.jpg}}
%\end{picture}



Physicist and Mathematician with an M.Sc. in Physics at the University of S\~ao Paulo (IFUSP). During my research project, I earned experience on data analysis with machine learning algorithms, while working on the simulation of novel materials for quantum computing.  Now, I intend to apply my abilities into the field of data science where I aim to make many contributions due to my outstanding analytic and mathematical skills. 

\section{Education }
\footnotetext[1]{\scriptsize{ Statistics from \url{https://planeacion.uniandes.edu.co/dmdocuments/Boletin_estadistico_2016.pdf}}}
\cventry{March 2017 -- March 2019 }{M.Sc. in Physics}{Institute of Physics, University of S\~ao Paulo (IFUSP)}{S\~ao Paulo, SP}{Brasil}{GPA: Maximum grade A.}

\cventry{Jan 2013 -- April 2017}{B.Sc. in Mathematics}{University of Los Andes}{Bogot\'a}{Colombia}{GPA: 4.45/5.00 . Class Rank: Percentile 90 \footnotemark[1].}

\cventry{Jan 2012 -- Oct 2016}{B.Sc. in Physics}{University of Los Andes}{Bogot\'a}{Colombia}{GPA: 4.45/ 5.00. Class Rank: Percentile 90 \footnotemark[1].}


\section{Programming}


\subsection{ Languages}
\cvitem{}{ Python, C++ , Octave, R, Java, SQL, Unix Shell Scripts}

\subsection{Software}
\cvitem{}{  Jupyter, Mathematica, Matlab, Rstudio, Eclipse,  Inkscape}

\subsection{Operating Systems}
\cvitem{}{ MacOs, Ubuntu, Windows. }

\subsection{Computational Experience }
\cventry{March 2017- May 2019}{Numerical simulation of basic quantum computing architectures}{Institute of Physics (USP)}{S\~ao Paulo}{Brasil}{}
\cvitem{}{\textbf{Relevant Experience }}
\cvitem{$\circ$}{Design of graph algorithms to simulate quantum transport.}
\cvitem{$\circ$}{Simulation of quantum materials with the Numerical Renormalization Group  on C++.}
\cvitem{$\circ$}{Integration of machine learning techniques to the study of quantum materials.}
\cvitem{$\circ$}{Data analysis with python. }
\cventry{Github: }{\url{ https://github.com/cifu9502}}{}{}{}{}

\subsection{Courses}
\cventry{June, 2019}{SQL for Data Science on Coursera}{Universisy of California}{Davis}{}{In process}
\cventry{June, 2019}{Machine Learning by Stanford University on Coursera.}{}{}{}{\url{https://www.coursera.org/account/accomplishments/certificate/WWHCFW93CT95}}
\cventry{Jan 2019}{Numerical Renormalization Group (NRG UFU/2019)}{University of Uberl\^andia}{Brasil}{}{}
\cventry{April 2018}{Course on Tensor Networks and Applications}{Institute of Physics USP}{S\~ao Paulo}{Brasil}{}
\cventry{Sept 2017}{Course on Machine Learning for Many-Body Physics}{ICTP-SAIFR}{S\~ao Paulo}{Brasil}{}
\cventry{Jan 2015 - May 2015}{Computational methods on science}{University of Los Andes}{Bogot\'a}{Colombia}{\footnotesize Grade: 4.46/5.0}
\cventry{Jan 2012 - Dec 2012}{Object Oriented Programming I y II}{University of Los Andes}{Bogot\'a}{Colombia}{\footnotesize Grade: 5.0/5.0}



\section{Upcoming Publication}
\cventry{May 2019}{Manipulating Majorana zero-modes in double quantum dots}{\url{https://arxiv.org/abs/1905.09140}}{Cifuentes, J. D. \& da Silva, L. G. G. V. D}{}{}


\section{Research Experience}
 
 
\cventry{March 2017 -- May 2019  }{Master's Thesis: "Kondo-Majorana Coupling in Double Quantum Dots "}{Advisor: Luis G. G. V. Dias da Silva.}{}{}{}{\hspace{2.5cm}{
\footnotesize  Link: \url{http://www.teses.usp.br/teses/disponiveis/43/43134/tde-22052019-115305/.}}
\vspace{2mm}

\cvitem{}{\textbf{Relevant Experience }}
\cvitem{$\circ$}{Developed efficient strategies for mathematical modeling in physics.}
\cvitem{$\circ$}{Team work and collaborative research with colleagues and professors from several universities in Brazil.}
\cvitem{$\circ$}{Acquired independent research skills and independent learning techniques. }
\cvitem{$\circ$}{Skillful writing for academic publications. }

\vspace{2mm}



%{ {\footnotesize Poster: \url{https://git.io/fp2vS}  }  \hspace{3.5cm} {\footnotesize Paper draft coming in a few weeks }}}
%{\footnotesize Poster: \url{https://git.io/fp2vS}  }}



\cventry{Aug 2016 - Dec 2016}{Undergraduate thesis in Mathematics: "On Gauging Symmetries of 2D Topological Quantum Field Theories" }{Advisor: C\'esar Galindo}{}{}{}{\hspace{2.5cm}
\footnotesize Link: \url{https://repositorio.uniandes.edu.co/handle/1992/15220} \hspace{1cm} Grade: 5.00/5.00.}


\cventry{Sept 2015 - May 2016}{Undergraduate thesis in Physics: "Understanding Topological order in 1D fermion chains"}{Advisor: Andr\'es Reyes Lega}{}{}{}{\hspace{2.5cm} \footnotesize Link: \url{https://repositorio.uniandes.edu.co/handle/1992/17840} \hspace{1cm} Grade: 4.80/5.00.} 




%\section{Transferable Skills}
%\begin{multicols}{2}

%\cvitem{}{Team work}
%\cvitem{}{Independent research}
%\cvitem{}{Data analysis}
%\cvitem{}{Coding skills}
%\cvitem{}{Object-oriented programming}
%\cvitem{}{Mathematical modeling}
%\cvitem{}{Independent learning}
%\cvitem{}{Academic writing}
%\cvitem{}{Machine learning}


%\end{multicols}



%\section{Research Interests}

%\begin{multicols}{2}
%\cvlistitem{Quantum Computation}
%\cvlistitem{Topological States of Matter}

%\cvlistitem{Quantum Information Theory}
%\end{multicols}


%\begin{footnotesize}
%Important courses and seminaries:
%\begin{enumerate}
%\item Workshop on Topological States of Matter
%\item Workshop on Topological Quantum Matter from Theory to Applications
%\item Course in Information Theory for Physicist
%\end{enumerate}
%\end{footnotesize}


\section{Awards}
\cvitem{March 2017 -- March 2019}{\textbf{Scholarship(CNPq)} for Master studies in Physics at the University of S\~ao Paulo.}

\cvitem{May 2012}{First place at \textbf{National Undergraduate Math Olympiad},  Colombia.}
\cvitem{July 2011}{Participant of the \textbf{52$^{th}$ International Mathematical Olympiad (IMO)}, Netherlands.  }
%\cvitem{2011}{First place at \textbf{National Physics Olympiad} (High school),  Colombia}

\section{Languages}

           \cvitem{Spanish}{Mother tongue}
           \cvitem{English}{Fluent. Toefl iBT: 101.  {\footnotesize Reading: 28 , Listening: 30, Speaking: 22 , Writing: 21}} 
           \cvitem{Portuguese}{Fluent.}

\vspace{2mm}

%----------------------------------------------------------------------------------------
%	WORK EXPERIENCE SECTION
%----------------------------------------------------------------------------------------

\section{Teaching Experience}
\cventry{2014, 2016}{Teaching Assistant of Linear Algebra I}{Department of Mathematics}{Universidad de los Andes}{Bogot\'a}{}{}
%\cventry{2015}{Co-founder of the Student Seminar on Theoretical Physics}{Universidad de los Andes}{Bogot\'a}{}{}
\cventry{2013, 2015}{Teaching Practice in Physics}{Department of Physics}{Universidad de los Andes}{Bogot\'a}{}{}
\cventry{June 2012, 2013, 2014}{Instructor at Summer Intership in High School Mathematical Olympiads}{Olimpiadas Colombianas de Matem\'aticas}{Universidad Antonio Nari\~no}{Bogot\'a}{}

\cvitem{}{\textbf{Relevant experience }}
\cvitem{$\circ$}{Capacity to summarize information and present it in clear, graphic and concise way}
\cvitem{$\circ$}{Experience with professional presentations }

% \section{Volunteer Work}
% \cventry{Aug 2015 -- Nov 2016}{Organizer of the Physics Seminar for Students}{}{Universidad de los Andes}{Bogot\'a.}{Annual part-time.}{}
% \cventry{Aug 2016 -- Sept 2016}{Peace lectures on Public Space}{}{Universidad de los Andes}{Bogot\'a}{Part-time.}{}
% \cventry{July 2013}{ Leading Guide at the 54th International Mathematical Olympiad }{Olimpiadas Colombianas de Matem\'aticas}{Santa Martha}{Colombia}{July 18 - July 28 . Full-time.}{}



%----------------------------------------------------------------------------------------
%	AWARDS SECTION
%----------------------------------------------------------------------------------------


%----------------------------------------------------------------------------------------
%	COMPUTER SKILLS SECTION
%----------------------------------------------------------------------------------------

%\section{Skills}
%\vspace{1mm}

%\subsection{Physics tests}
%           \cvitem{GRE Subject in Physics}{ Scaled score: 820 , Percentile: 74}
 %          \cvitem{EUF}{Score: 7.9 , Classification: 4 }
%\vspace{1mm}




%\cventry{May 2018}{ Autumn Meeting of the Brazilian Physical Society}{Foz do Igu\c{c} }{Brasil}{Poster in: Edge-States and Topological Phase Transitions in the Kitaev Chain}






%----------------------------------------------------------------------------------------
%	INTERESTS SECTION
%----------------------------------------------------------------------------------------


%----------------------------------------------------------------------------------------
%	COVER LETTER
%----------------------------------------------------------------------------------------

% To remove the cover letter, comment out this entire block

%\clearpage

%\recipient{Organizing Comitee}{Villa de Leyva Summer School 2015} % Letter recipient
%\date{\today} % Letter date
%\opening{Dear Organizing Comitee,} % Opening greeting
%\closing{Sincerely,} % Closing phrase
%\enclosure[Attached]{curriculum vit\ae{}} % List of enclosed documents

%\makelettertitle % Print letter title

%\begin{small}

%\end{small}

%\makeletterclosing % Print letter signature

%----------------------------------------------------------------------------------------

\end{document}